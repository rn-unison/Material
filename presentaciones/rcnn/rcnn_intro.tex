% Created 2018-10-08 Mon 08:57
% Intended LaTeX compiler: pdflatex
\documentclass[presentation, t]{beamer}
\usepackage[utf8]{inputenc}
\usepackage[T1]{fontenc}
\usepackage{graphicx}
\usepackage{grffile}
\usepackage{longtable}
\usepackage{wrapfig}
\usepackage{rotating}
\usepackage[normalem]{ulem}
\usepackage{amsmath}
\usepackage{textcomp}
\usepackage{amssymb}
\usepackage{capt-of}
\usepackage{hyperref}
\setbeamertemplate{navigation symbols}{}
% typeset source code listings
\usepackage{listings}
\usepackage{xcolor}
\definecolor{mc@lstidentifier}{HTML}{000000} % black
\definecolor{mc@lstcomment}{HTML}{008200} % green
\definecolor{mc@lststring}{HTML}{FF5500} % orange
\definecolor{mc@lstkeyword}{HTML}{0000FF} % blue
\definecolor{mc@lstbackground}{HTML}{FFFFCC} % light yellow
\definecolor{mc@lstframe}{HTML}{FFEE88} % dark yellow
\lstdefinelanguage{org}{%
morekeywords={:results, :session, :var, :noweb, :exports},
sensitive=false,
morestring=[b]",
morecomment=[l]{\#},
}
\lstset{%
lineskip=-0.1em,
%
basicstyle=\ttfamily\scriptsize, % font that is used for the code
identifierstyle=\color{mc@lstidentifier},
commentstyle=\color{mc@lstcomment}\itshape,
stringstyle=\color{mc@lststring},
keywordstyle=\color{mc@lstkeyword},
%
extendedchars=true,
inputencoding=utf8,
upquote, %
tabsize=4, % set default tabsize to 4 spaces
showtabs=false, % show tabs within strings adding particular underscores
%  tab=$\to$,
showspaces=false, % show spaces adding particular underscores
showstringspaces=false, % underline spaces within strings
%
numbers=left, % where to put the line numbers
stepnumber=0, % step between two line numbers
numberstyle=\tiny, % line number font size
%
captionpos=b, % set the caption position to `bottom'
%
xleftmargin=0.4em, % text to the right
xrightmargin=0.4em, % text to the left
breaklines=false, % don't break long lines of code
%
frame=single, % add a frame around the code
framexleftmargin=0pt, % frame back to the left
framexrightmargin=0pt, % frame back to the right
backgroundcolor=\color{mc@lstbackground}, % set the background color
rulecolor=\color{mc@lstframe}, % frame color
%
columns=flexible, % try not to ruin the spacing intended by the font designer
keepspaces=true, % don't drop spaces to fix column alignment
%
mathescape, % allow escaping to (La)TeX mode within $..$
% escapechar=², % allow escaping to (La)TeX mode within ²..²
% The backquote was NOT judicious: in some code (comments), I wrap vars
% between such a backquote (`var')
%
% conversion of UTF-8 chars to latin1
literate=
{á}{{\'a}}1
{à}{{\`a}}1
{â}{{\^a}}1
{ä}{{\"a}}1
{ç}{{\c{c}}}1
{é}{{\color{black}\'e}}1
{è}{{\`e}}1
{ê}{{\^e}}1
{ë}{{\"e}}1
{í}{{\'i}}1
{ì}{{\`i}}1
{î}{{\^i}}1
{ï}{{\"i}}1
{ó}{{\'o}}1
{ò}{{\`o}}1
{ô}{{\^o}}1
{ö}{{\"o}}1
{ú}{{\'u}}1
{ù}{{\`u}}1
{û}{{\^u}}1
{ü}{{\"u}}1
{Á}{{\'A}}1
{À}{{\`A}}1
{Â}{{\^A}}1
{Ä}{{\"A}}1
{Ç}{{\c{C}}}1
{É}{{\'E}}1
{È}{{\`E}}1
{Ê}{{\^E}}1
{Ë}{{\"E}}1
{Í}{{\'I}}1
{Ì}{{\`I}}1
{Î}{{\^I}}1
{Ï}{{\"I}}1
{Ó}{{\'O}}1
{Ò}{{\`O}}1
{Ô}{{\^O}}1
{Ö}{{\"O}}1
{Ú}{{\'U}}1
{Ù}{{\`U}}1
{Û}{{\^U}}1
{Ü}{{\"U}}1
}
\definecolor{TeXbackgroundcolor}{HTML}{F1F9EF}
\definecolor{TeXrulecolor}{HTML}{D4E8E3}
\lstdefinestyle{TeX}{backgroundcolor=\color{TeXbackgroundcolor},rulecolor=\color{TeXrulecolor}}
\definecolor{mylinkcolor}{HTML}{006DAF}
\hypersetup{colorlinks=true, linkcolor=mylinkcolor, urlcolor=mylinkcolor}
\usetheme{Madrid}
\author{\href{mailto:julio.waissman@unison.mx}{Julio Waissman}}
\date{12 de abril de 2018}
\title{Una revisión sobre redes convolucionales para localización de objetos en imágenes}
\institute[Unison]{Licenciatura en Ciencias de la Computación\\Universidad de Sonora}
\titlegraphic{\includegraphics[width=\textwidth,height=.3\textheight]{portada.png}}
\usepackage[spanish]{babel}
\usepackage[spanish]{bm}
\hypersetup{
 pdfauthor={\href{mailto:julio.waissman@unison.mx}{Julio Waissman}},
 pdftitle={Una revisión sobre redes convolucionales para localización de objetos en imágenes},
 pdfkeywords={},
 pdfsubject={Presentación para la SIDM},
 pdfcreator={Emacs 25.1.50.2 (Org mode 9.1.14)},
 pdflang={Spanish}}
\begin{document}

\maketitle


\begin{frame}[label={sec:orgbc0108b}]{Plan de la presentación}
\begin{enumerate}
\item ¿Qué es la localización de objetos en imágenes?
\item Arquitectura RCNN
\begin{itemize}
\item La arquitectura básica: RCNN
\item Su primer modificación: Fast RCNN
\item La que se emplea actualmente: Fastest RCNN\vfill
\end{itemize}
\item Arquitecturas alternativas
\begin{itemize}
\item R-FCN
\item SSD\vfill
\end{itemize}
\item Conclusiones\vfill
\end{enumerate}
\end{frame}



\begin{frame}[label={sec:org124ae9e}]{Detección de objetos en imágenes}
\begin{enumerate}
\item Redes convolucionales para tratamiento de imágenes\vfill

\item Reconocimiento de objetos en imágenes, la aplicación más exitosa de
las CNN\vfill

\item Localización de objetos en imágenes es parecido a aplicar el
reconocimiento en muchas subimágenes y luego fusionar la
información\vfill
\end{enumerate}
\end{frame}





\begin{frame}[label={sec:org6b47e45}]{Redes convolucionales basadas en regiones (RCNN)}
\begin{enumerate}
\item La primera red convolucional para detección de objetos

\item Es la base de todas las redes siguientes, pero solo se utiliza
actualmente como método ilustrativo.

\item Muy lenta, implica el entrenamiento de muchos métodos
\end{enumerate}
\end{frame}



\begin{frame}[label={sec:org2ad81c0}]{Redes convolucionales basadas en regiones (RCNN)}
\framesubtitle{Algortimo general}

\begin{enumerate}
\item Escanea la imágen de entrada utilizando un algoritmo llamado \emph{Selective Search},
y genera al rededor de 2000 regones para probar si en éstas se encuentra algun objeto.

\item Ejecuta una red convolucional (capas convolucionales y de \emph{max pooling} por cada una de las regiones.

\item Toma la salida de cada una de dichas CNNs.
\end{enumerate}
\end{frame}


\begin{frame}[label={sec:org66818a0}]{Redes convolucionales basadas en regiones (RCNN)}
\vfill
\begin{figure}[htbp]
\centering
\includegraphics[width=0.99\textwidth]{./imagenes/rcnn.png}
\end{figure}\vfill
\end{frame}


\begin{frame}[label={sec:orgb7b11da}]{Redes convolucionales basadas en regiones mejorada (Fast-RCNN)}
\vfill

\begin{figure}[htbp]
\centering
\includegraphics[width=0.90\textwidth]{./imagenes/fast-rcnn.png}
\end{figure}\vfill
\end{frame}

\begin{frame}[label={sec:org95e241e}]{Última version mejorada (Fast-RCNN)}
\vfill

\begin{figure}[htbp]
\centering
\includegraphics[width=0.70\textwidth]{./imagenes/faster-rcnn.png}
\end{figure}\vfill
\end{frame}

\begin{frame}[label={sec:orgf6447f9}]{Otra arquitectura no tan directa (RFCN)}
\vfill

\begin{figure}[htbp]
\centering
\includegraphics[width=0.90\textwidth]{./imagenes/rfcn.png}
\end{figure}\vfill
\end{frame}

\begin{frame}[label={sec:org2a9f368}]{Ejemplo de RFCN}
\vfill

\begin{figure}[htbp]
\centering
\includegraphics[width=0.70\textwidth]{./imagenes/rfcn-ejemplo.png}
\end{figure}\vfill
\end{frame}

\begin{frame}[label={sec:orgc71bec6}]{Otra arquitectura más (SSD)}
\vfill

\begin{figure}[htbp]
\centering
\includegraphics[width=0.90\textwidth]{./imagenes/ssd.png}
\end{figure}\vfill
\end{frame}

\begin{frame}[label={sec:orgcb77c59}]{Ejemplo de SSD}
\vfill

\begin{figure}[htbp]
\centering
\includegraphics[width=0.90\textwidth]{./imagenes/ssd-ejemplo.png}
\end{figure}\vfill
\end{frame}
\end{document}